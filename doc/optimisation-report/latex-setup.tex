% This is a simple template for a LaTeX document using the "article" class.
% See "book", "report", "letter" for other types of document.

% use larger type; default would be 10pt
\documentclass[11pt]{scrartcl}

% set input encoding (not needed with XeLaTeX)
\usepackage[utf8]{inputenc}
\usepackage[title,titletoc,header]{appendix}
% \usepackage[toc,page]{appendix}
\usepackage{multicol}
% \usepackage{titling}
\usepackage{longtable}

\usepackage{booktabs} % for much better looking tables
\usepackage{tabularx}

\usepackage{amsmath}
\usepackage{amssymb}
\usepackage{hyperref}
% \usepackage{url}

\usepackage{todonotes}
% \usepackage[disable]{todonotes}
\presetkeys{todonotes}{inline}{}

\usepackage{natbib}
% To cite, use \citet{} in text citation, or \citep{} (in parentheses).

%%% PAGE DIMENSIONS
\usepackage[a4paper]{geometry} % to change the page dimensions
% \geometry{a4paper, margin=1.3in} % or letterpaper (US) or a5paper or....
% \geometry{margin=2in} % for example, change the margins to 2 inches all round
% \geometry{landscape} % set up the page for landscape
%   read geometry.pdf for detailed page layout information

\usepackage{graphicx} 				% support the \includegraphics command and options

% \usepackage[parfill]{parskip} % Activate to begin paragraphs with an empty line rather than an indent

%%% PACKAGES
% \usepackage{booktabs} 				% for much better looking tables
\usepackage{array} 					% for better arrays (eg matrices) in maths
\usepackage{paralist} 				% very flexible & customisable lists (eg. enumerate/itemize, etc.)
\usepackage{verbatim} 				% adds environment for commenting out blocks of text & for better verbatim
\usepackage{subfig} 				% make it possible to include more than one captioned figure/table in a single float
\usepackage{float} 					% allow floating for figures

\usepackage{pgfgantt} 				% gantt charts
% \usepackage[export]{adjustbox}[2011/08/13] % For centering wide figures

\usepackage[binary-units]{siunitx}            % For SI units
\usepackage{rotating}
\usepackage{pdflscape}
\usepackage{enumitem} % control layout of itemize, enumerate, description
\usepackage{pseudocode}
\usepackage{mathtools}

%%% PARAGRAPHING & INDENTATION
%\setlength{\parindent}{0pt}
%\setlength{\parskip}{2ex plus 0.5ex minus 0.3ex}

%%% HEADERS & FOOTERS
\usepackage{fancyhdr} 				% This should be set AFTER setting up the page geometry
\pagestyle{fancy} 					% options: empty , plain , fancy
\renewcommand{\headrulewidth}{0pt} 	% customise the layout...
\lhead{}\chead{}\rhead{}
\lfoot{}\cfoot{\thepage}\rfoot{}

%%% SECTION TITLE APPEARANCE
\usepackage{sectsty}
\allsectionsfont{\sffamily\mdseries\upshape} % (See the fntguide.pdf for font help)
% (This matches ConTeXt defaults)

%%%% ABSTRACT APPEARANCE
%\usepackage{abstract}
%\renewcommand{\absnamepos}{flushleft}
%\setlength{\absleftindent}{0pt}
%\setlength{\absrightindent}{0pt}

%%% TABLE OF CONTENTS APPEARANCE
 % Put the bibliography in the ToC
\usepackage[nottoc,notlof,notlot]{tocbibind}
 % Alter the style of the Table of Contents
\usepackage[titles,subfigure]{tocloft}
\renewcommand{\cftsecfont}{\rmfamily\mdseries\upshape}
 % No bold!
\renewcommand{\cftsecpagefont}{\rmfamily\mdseries\upshape}

%%% REFERENCES APPEARANCE
% \renewcommand{\bibname}{References}

\newcommand{\code}[1]{{\texttt{#1}}}
\newcommand{\libraryname}[1]{{\texttt{#1}}}
\newcommand{\codefile}[1]{{\textit{#1}}}
\newcommand{\program}[1]{\code{#1}}
\newcommand{\taskname}[1]{{\textit{#1}}}
\newcommand{\newterm}[1]{{\textit{#1}}}
\newcommand{\scarequotes}[1]{`#1'}
