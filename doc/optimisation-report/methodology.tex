\subsection{Metaheuristics considered} {
    \subsubsection{Metaheuristics considered} {
        This work compares four metaheuristics:
        \begin{itemize}
            \item Simulated Annealing \citep{kirkpatrick:1983op, vcerny:1985th};
            \item Iterated tabu search \citep{Misevicius:2012dj};
            \item BMA \citep{Benlic:2015gp};
            \item A simple evolutionary algorithm.
        \end{itemize}
    }

    \subsubsection{Iterated tabu search (ITS) \citep{Misevicius:2012dj}} {
        Alternates two steps:
        \begin{enumerate}
            \item \newterm{Controlled chained mutation}
                \begin{itemize}
                    \item Performs a \newterm{chained mutation}
                    \item Chooses the most \newterm{disruptive} mutation from a set
                    \item Controls mutation size and disruptiveness
                \end{itemize}
            \item \newterm{Improved robust tabu-search}
                Tabu search with extra rules to deter \scarequotes{stagnant behaviour}:
                \begin{itemize}
                    \item periodically performs steepest descent search
                    \item periodically ignores the tabu-list
                    \item halves all tabu-counts when a new local optimum is reached
                \end{itemize}
        \end{enumerate}
    }

    \subsubsection{BMA \citep{Benlic:2015gp}} {
        BMA has the following features:
        \begin{description}
            \item[Local search:] Breakout local search (BLS) \citep{Benlic:2013gi};
            \item[Crossover:] \scarequotes{The} uniform crossover (UX) operator;
            \item[Mutation:] Chained sequence mutation.
        \end{description}
    }

    \subsubsection{Breakout local search (BLS)} {
        `Breakout local search for the quadratic assignment problem' \citep{Benlic:2013gi}.

        Each iteration:
        \begin{enumerate}
            \item Perform steepest descent search using a 2-swap neighbourhood.
            \item Perform a number of perturbation moves:
                \begin{itemize}
                    \item Either random moves or tabu search moves;
                    \item Perturbation type chosen based on last improving iteration;
                    \item Number of moves increases with visits to the same local optimum;
                \end{itemize}
        \end{enumerate}
    }

    \subsubsection{Evolutionary algorithm} {
        An evolutionary algorithm without a local improvement method:
        \begin{itemize}
            \item Maintains a population of \(N\) solutions;
            \item Each iteration:
            \begin{enumerate}
                \item Remove all but the \(K\) best solutions from the population.
                \item Generate \(N-K\) new individuals using either crossover or mutation, chosen randomly:
                    \begin{itemize}
                        \item Chained sequence mutation
                        \item Uniform crossover
                    \end{itemize}
            \end{enumerate}
        \end{itemize}
    }
}

\subsection{Method of comparison} {
    \subsubsection{Experiment} {
        The performance of the algorithms was compared on the full set of QAPLIB problems \citep{Burkard:1997ve}.

        \begin{itemize}
            \item Ran each algorithm on all 135 of the 136 problem instances (the trivial instance \texttt{esc16f} was excluded).
            \item Time limit of \SI{5}{\second} per run.
            \item Record best solution, time that best solution was found, and actual time taken.
            \item Test significance of performance difference using Wilcoxon signed rank tests.
        \end{itemize}
    }
}
