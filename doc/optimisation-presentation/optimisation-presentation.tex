
% This is a simple template for a LaTeX document using the "article" class.
% See "book", "report", "letter" for other types of document.

% use larger type; default would be 10pt
\documentclass[11pt]{scrartcl}

% set input encoding (not needed with XeLaTeX)
\usepackage[utf8]{inputenc}
\usepackage[title,titletoc,header]{appendix}
% \usepackage[toc,page]{appendix}
\usepackage{multicol}
% \usepackage{titling}
\usepackage{longtable}

\usepackage{booktabs} % for much better looking tables
\usepackage{tabularx}

\usepackage{amsmath}
\usepackage{amssymb}
\usepackage{hyperref}
% \usepackage{url}

\usepackage{todonotes}
% \usepackage[disable]{todonotes}
\presetkeys{todonotes}{inline}{}

\usepackage{natbib}
% To cite, use \citet{} in text citation, or \citep{} (in parentheses).

%%% PAGE DIMENSIONS
\usepackage[a4paper]{geometry} % to change the page dimensions
% \geometry{a4paper, margin=1.3in} % or letterpaper (US) or a5paper or....
% \geometry{margin=2in} % for example, change the margins to 2 inches all round
% \geometry{landscape} % set up the page for landscape
%   read geometry.pdf for detailed page layout information

\usepackage{graphicx} 				% support the \includegraphics command and options

% \usepackage[parfill]{parskip} % Activate to begin paragraphs with an empty line rather than an indent

%%% PACKAGES
% \usepackage{booktabs} 				% for much better looking tables
\usepackage{array} 					% for better arrays (eg matrices) in maths
\usepackage{paralist} 				% very flexible & customisable lists (eg. enumerate/itemize, etc.)
\usepackage{verbatim} 				% adds environment for commenting out blocks of text & for better verbatim
\usepackage{subfig} 				% make it possible to include more than one captioned figure/table in a single float
\usepackage{float} 					% allow floating for figures

\usepackage{pgfgantt} 				% gantt charts
% \usepackage[export]{adjustbox}[2011/08/13] % For centering wide figures

\usepackage[binary-units]{siunitx}            % For SI units
\usepackage{rotating}
\usepackage{pdflscape}
\usepackage{enumitem} % control layout of itemize, enumerate, description
\usepackage{pseudocode}
\usepackage{mathtools}

%%% PARAGRAPHING & INDENTATION
%\setlength{\parindent}{0pt}
%\setlength{\parskip}{2ex plus 0.5ex minus 0.3ex}

%%% HEADERS & FOOTERS
\usepackage{fancyhdr} 				% This should be set AFTER setting up the page geometry
\pagestyle{fancy} 					% options: empty , plain , fancy
\renewcommand{\headrulewidth}{0pt} 	% customise the layout...
\lhead{}\chead{}\rhead{}
\lfoot{}\cfoot{\thepage}\rfoot{}

%%% SECTION TITLE APPEARANCE
\usepackage{sectsty}
\allsectionsfont{\sffamily\mdseries\upshape} % (See the fntguide.pdf for font help)
% (This matches ConTeXt defaults)

%%%% ABSTRACT APPEARANCE
%\usepackage{abstract}
%\renewcommand{\absnamepos}{flushleft}
%\setlength{\absleftindent}{0pt}
%\setlength{\absrightindent}{0pt}

%%% TABLE OF CONTENTS APPEARANCE
 % Put the bibliography in the ToC
\usepackage[nottoc,notlof,notlot]{tocbibind}
 % Alter the style of the Table of Contents
\usepackage[titles,subfigure]{tocloft}
\renewcommand{\cftsecfont}{\rmfamily\mdseries\upshape}
 % No bold!
\renewcommand{\cftsecpagefont}{\rmfamily\mdseries\upshape}

%%% REFERENCES APPEARANCE
% \renewcommand{\bibname}{References}

\newcommand{\code}[1]{{\texttt{#1}}}
\newcommand{\libraryname}[1]{{\texttt{#1}}}
\newcommand{\codefile}[1]{{\textit{#1}}}
\newcommand{\program}[1]{\code{#1}}
\newcommand{\taskname}[1]{{\textit{#1}}}
\newcommand{\newterm}[1]{{\textit{#1}}}
\newcommand{\scarequotes}[1]{`#1'}


% \title{Optimisation of ball handling behaviour in humanoid robot soccer}
% \title{\vspace{-2.0cm}Biped walk stability improvement for a small humanoid robot}
% \subtitle{COMP4120 research presentation}
% \author{Mitchell Metcalfe}

\title{Comparing metaheuristic approaches to solving the quadratic assignment problem}
\subtitle{COMP4240 project presentation}
\author{Mitchell Metcalfe}
\institute{The University of Newcastle, Australia}
% \date{\today \\[1.5\baselineskip] Supervised by Dr. Alex Mendes}
% \date{\today \\[1.5\baselineskip] Supervised by Dr. Alex Mendes}

% Activate to display a given date or no date (if empty), otherwise the current date is printed
% \date{\today}
% \date{May 30, 2015}

% \rhead{COMP4120 research presentation, \today}

\begin{document}

\maketitle

\begin{frame}{Outline}
    \tableofcontents
\end{frame}

% a) an introduction to the topic studied
\section{Background and motivation} {
    \subsection{The quadratic assignment problem (QAP)} {
        \begin{frame}{The quadratic assignment problem (QAP)}
            \centering

            Introduced by \citet{Koopmans:1957gf}

            \vspace{1cm}

            `Assignment Problems and the Location of Economic Activities'
        \end{frame}

        \begin{frame}{The quadratic assignment problem (QAP)}
            Assign manufacturing plants to locations in a way that maximises total revenue.

            Account for the following complicating factors:
            \begin{itemize}
                \item The revenue of each plant is dependent on its location;
                \item Pairs of plants must transport a given number of commodity bundles between them per unit time;
                \item Transportation cost is proportional to distance.
            \end{itemize}
        \end{frame}

        \begin{frame}{Mathematical formulation}
            Given the matrices:
            \begin{itemize}
                \item \([r_{ki}]\): revenue of plant \(k\) at location \(i\)
                \item \([a_{kl}]\): required commodity \newterm{flow} between plants \(k\) and \(l\)
                \item \([b_{ij}]\): cost of transport per unit flow between locations \(i\) and \(j\)
            \end{itemize}

            Find a permutation \(\pi^{*} \in S_n\) that maximises the total revenue:

            \[ \pi^{*} = \max_{\pi} \left(\sum_{k}{r_{k\pi(k)}} - \sum_{k}\sum_{l}{a_{kl}b_{\pi(k)\pi(l)}}\right) \]

            where  \(\pi(k) = i\) indicates that plant \(k\) is to be placed at location \(i\).
        \end{frame}

        \begin{frame}{Minimisation problem}
            Given the matrices:
            \begin{itemize}
                \item \([a_{kl}]\): required commodity \newterm{flow} between plants \(k\) and \(l\)
                \item \([b_{ij}]\): cost of transport per unit flow between locations \(i\) and \(j\)
            \end{itemize}

            Find a permutation \(\pi^{*} \in S_n\) that minimises the transportation cost:

            \[ \pi^{*} = \min_{\pi} \sum_{k}\sum_{l}{a_{kl}b_{\pi(k)\pi(l)}} \]

            where  \(\pi(k) = i\) indicates that plant \(k\) is to be placed at location \(i\).
        \end{frame}

        \begin{frame}{Difficulty}
            \begin{itemize}
                \item NP-Hard.
                \item Contains the Travelling Salesman Problem as a special case:
            \end{itemize}

            \todo{Include the expression of the TSP \citep{Merz:2000vr}}

            \vspace{0.6cm}

            \begin{center}
                ``one of the most difficult problems in the NP-hard class'' \citep{Loiola:2007jk}
            \end{center}
        \end{frame}
    }

    \subsection{Recent applications} {
        \begin{frame}{Applications of the QAP}
            The facilities layout problem (FLP):
            \begin{itemize}
                \item Optimally locate manufacturing plants to maximise revenue.
                \item Used by \citeauthor{Koopmans:1957gf} to motivate the QAP.
                \item The most common application of the QAP \citep{Loiola:2007jk}.
            \end{itemize}

            \citet{Loiola:2007jk} surveys 365 papers published between 1957 and 2007.
        \end{frame}

        \begin{frame}{Recent FLP applications}
            \begin{description}
                \item[\citet{Samanta:2015hk}:] Layout optimisation of a bus body manufacturing plant.
                \item[\citet{XiongfengFeng:2015jo}:] Layout of departments in a hospital.
                % Decreased the average walking time for outpatients by 11.55\%.
            \end{description}
        \end{frame}

        \begin{frame}{Other recent applications}
            \begin{description}
                \item[\citet{Alguliyev:2015jw}:] Unsupervised document summarisation.
                \item[\citet{Azab:2015eq}:] Machine features of a product in an optimal sequence in order to minimise handling time, given a set of precedence constraints between features.
            \end{description}
        \end{frame}
    }
}

\section{The QAP in the literature} {
    \begin{frame}{Recent literature}
        % Include 5 papers:
        Memory schemas \citet{Meneses:2011hg}
        \citet{Harris:2015kw}
        Particle swarm optimisation and Tabu search \citet{Helal:2015de}
    \end{frame}
}

\section{Comparison of metaheuristics} {
    \subsection{Metaheuristics considered} {
        \begin{frame}{Metaheuristics considered}
            This work compares four metaheuristics:
            \begin{itemize}
                \item Simulated Annealing \citep{kirkpatrick:1983op, vcerny:1985th};
                \item Iterated tabu search \citep{Misevicius:2012dj};
                \item BMA \citep{Benlic:2015gp};
                \item A simple evolutionary algorithm.
            \end{itemize}
        \end{frame}

        \begin{frame}{Iterated tabu search (ITS) \citep{Misevicius:2012dj}}

        \end{frame}

        \begin{frame}{BMA \citep{Benlic:2015gp}}
            \begin{block}{Memetic algorithm \citep{Neri:2012jr}}
                A metaheuristic combining a population-based approach with a local improvement method.
            \end{block}

            \vspace{0.8cm}

            BMA has the following features:
            \begin{description}
                \item[Local search:] Breakout local search (BLS) \citep{Benlic:2013gi};
                \item[Crossover:] \scarequotes{The} uniform crossover (UX) operator;
                \item[Mutation:] Chained sequence mutation.
            \end{description}
        \end{frame}

        \begin{frame}{Breakout local search (BLS)}
            `Breakout local search for the quadratic assignment problem' \citep{Benlic:2013gi}.

            \begin{enumerate}
                \item Perform steepest descent search using a 2-swap neighbourhood.
                \item Perform a number of perturbation moves:
                    \begin{itemize}
                        \item Either random moves or tabu search moves;
                        \item Perturbation type chosen based on last improving iteration;
                        \item Number of moves increases with visits to the same local optimum;
                    \end{itemize}
            \end{enumerate}
        \end{frame}

        \begin{frame}{Evolutionary algorithm}

        \end{frame}
    }

    \subsection{Method of comparison} {
        \begin{frame}{Experiment}
            The performance of the algorithms was compared on the full set of QAPLIB problems \citep{Burkard:1997ve}.

            \begin{itemize}
                \item Ran each algorithm on all 135 of the 136 problem instances (the trivial instance \texttt{esc16f} was excluded).
                \item Time limit of \SI{5}{\second} per run.
                \item Record best solution, time that best solution was found, and actual time taken.
                \item Test significance of performance difference using a Friedman test.
            \end{itemize}
        \end{frame}
    }
}

\section{Preliminary results} {
    \begin{frame}{Experiment results}

    \end{frame}

    \begin{frame}{Significance test}

    \end{frame}

    \begin{frame}{Summary of results}

    \end{frame}
}

\section{Conclusion} {
    \begin{frame}{Conclusion}
        In summary, this work compared four metaheuristics based on recent approaches from the literature to solving the QAP.

        % TODO: Showed that X showed promise for future research.
        % TODO: The memetic algorithm outperformed the classical approaches.
        % TODO: Tabu search performed well.
    \end{frame}

    \begin{frame}{Bibliography}
      \vspace{-2em}
      \bibliographystyle{apalike}
    %   \scriptsize
      \fontsize{0.5em}{0.5em}\selectfont
      \bibliography{bibliography}
    \end{frame}

    \plain{Questions?}
}


% \begin{frame}{NUbots}
%     \includegraphics[height=0.6\textwidth]{figures/nubots-first-frame.png}
% \end{frame}
% \begin{frame}{ZMP Limitations}
%     \begin{columns}
%       \column{0.38\linewidth}
%          \centering
%          \includegraphics[width=\linewidth]{figures/robot-rotate.pdf}
%        \column{0.58\linewidth}
%        \begin{itemize}
%            \item Gives no useful information if the robot begins to rotate about the edge of a foot
%            \item Not defined if the point on the ground is not within the support polygon of the robot (as the ground reaction force only acts on the feet)
%            \begin{itemize}
%                \item Called the ‘fictitious ZMP’ or the ‘foot rotation indicator' \citep{Goswami:1999hd}
%                \item Can indicate the severity of unbalance, but not how to restore balance
%            \end{itemize}
%        \end{itemize}
%      \end{columns}
% \end{frame}


\end{document}
